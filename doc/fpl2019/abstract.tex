% As a general rule, do not put math, special symbols or citations
% in the abstract
\begin{abstract}
Prevalent hardware description languages, e.g., Verilog and VHDL, employ the register-transfer level (RTL) as their underlying programming model. A major downside of the RTL model is that it tightly couples design functionality with timing and device constraints. As a result, increasing design complexity results in code that is more verbose and less portable.
Emerging high-level synthesis (HLS) tools attempt to bridge this hardware programmability gap by utilizing constructs from imperative programming languages. These constructs and their sequential semantics, however, impede construction of inherently parallel hardware and data scheduling, which is crucial in many design use-cases.

In this paper we propose to use constructs from dataflow programming languages as basis for hardware design. We present DFiant, a Scala-embedded HDL that leverages dataflow semantics to decouple functionality from implementation constraints.
DFiant bridges the timing-agnostic and device-agnostic hardware description gaps by using the dataflow firing rule as a logical construct, coupled with modern software language features (e.g., inheritance, polymorphism) and classic HDL traits (e.g., bit-accuracy, input/output ports). Through DFiant, we demonstrate how dataflow constructs can be used to write code that is substantially more portable and compact than that of RTL and HLS languages.

We implemented a compiler for DFiant that transforms DFiant code into a dataflow graph, auto-pipelines the design to meet the target performance and device requirements, and maps the graph into synthesizable VHDL code. 
\end{abstract}


%For over two decades, register-transfer level (RTL) has been the dominant programming model as the basis for hardware description languages (HDLs) such as Verilog and VHDL. Unfortunately, RTL language-constructs tightly couple design functionality with timing and device constraints, thus as the design complexity grows the design code becomes more verbose and less portable.
%To bridge this hardware programmability gap, high-level synthesis (HLS) tools were introduced and based on software languages such as C. However, sequential software language semantics prevent designers from controlling hardware construction and data scheduling, which is crucial in many design use-cases. 

%In this paper we introduce a dataflow hardware programming model as the basis for DFiant, a Scala-embedded HDL that decouples functionality from implementation constraints. 
%DFiant bridges the timing-agnostic and device-agnostic hardware description gaps by incorporating dataflow semantics together with modern software language features such as inheritance and polymorphism; and typical HDL traits such as bit-accuracy, input/output ports, and component composition. DFiant is not an HLS language, nor is it an RTL language, but it is still an HDL that provides abstractions beyond RTL behavioral modeling, to reduce verbosity yet maintain portable codes. 

