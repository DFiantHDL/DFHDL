
\chapter{A main chapter}
\label{chap:firstchap}

\section{Introduction}

You might have a per-chapter mini-intro, possibly tying in to the relevant part of the general intro.

\section{A section}

\lipsum[1]

Let's cite a source: \cite{Yao1977}.

An equation...
\begin{align}
\label{eq:emc2}
e &= mc^2
\end{align}

In \autoref{sec:thm-like} below, we will state some theorems

\section{Results... and theorem-like environments}
\label{sec:thm-like}

What's so special about the theorem-like environments used here? There are several packages which offer the capability of defining these, mainly \texttt{amsthm}, \texttt{ntheorem} and also \texttt{thmtools}. (The last is probably also the most feature-full and versatile, but I'm not familiar with it and the first two are the popular ones.) Many people writing a Technion thesis start with \texttt{amsthm}, only to find out it has conflicts with Hebrew... also, there's the issue of aliasing (same counter for lemmata and theorems, but having \texttt{{\textbackslash}autoref} and similar commands know what they're referencing.) This is all neatly resolved in \texttt{iitthesis-extra.sty} with \texttt{amsthm}-like-looking environments actually done with nthrerom.

\begin{theorem}
\label{thm:first}
This is the first numbered theorem in this thesis.
\end{theorem}

And we can refer to it using \texttt{ref}: \ref{thm:first} and get the number, or use hypertex's \texttt{autoref}: \autoref{thm:first}.

\begin{corollary}
\label{cor:first}
There are no lemmata appearing before theorems in this thesis.
\end{corollary}

\begin{theorem*}
This is the second theorem, unnumbered.
\end{theorem*}

\begin{theorem*}[\protect{\cite[Theorem 2]{Knuth1973}}]
This is an unnumbered theorem cited from elsewhere. \qbfox{1} ... and it was Knuth's dog.
\end{theorem*}

\begin{note}
This is a note environment.  \qbfox{2}
\end{note}

\begin{definition}
\label{def:first}
An \emph{quick brown fox} is a fox which is not only fast and agile but is also characterized by brown fur. Such foxes sometimes tend to jump over lazy dogs.
\end{definition}

\begin{lemma}
\label{lem:first}
This is a lemma. \qbfox{2}
\end{lemma}

Even though \autoref{lem:first} and \autoref{def:first} share the ``same'' counter, when referring to them, their names are used automagically.

Here's a proof of the lemma:
\begin{proof}%[lem:natural:blowups-preserve-distance-on-average]
\lipsum[2]
\end{proof}

And here's a proof of \autoref{thm:first} using the \verb|proofof| environment.
\begin{proofof}[thm:first]
\lipsum[3]
\end{proofof}

\begin{proposition}
\label{prop:first}
A proposition environment. \qbfox{2}
\end{proposition}

\begin{observation}
\label{obs:first}
The moon revolves around the earth.
\end{observation}

There are several more environments of various kinds defined in \texttt{iitthesis-extra.sty}.

\subsection{A subsection}

We've started a subsection.

\begin{algorithm}
\caption{A nice algorithm}
\label{alg:first}
\begin{algorithmic}[1]
\FOR{$n$ times}
  \STATE{Do something.}
  \STATE{Do something else.}
\ENDFOR
\STATE{And do one last thing.}
\end{algorithmic}
\end{algorithm}

It is recommended to use \texttt{algorithmicx} over \texttt{algorithmic} for algorithms, like in \autoref{alg:first}, as it has less conflicts with Hebrew babel (regardless of whether you have Hebrew in your algorithms or not). Also, \texttt{iitthesis-extra.sty} provides it with a necessary workaround.

\subsection{A second subsection}

In this subsection we'll have a figure.

\begin{figure}[htb]
  \centering
%  \includegraphics{graphics/mygraphic1.pdf}
  \caption{Two circles and a wavy line.}
\end{figure}

