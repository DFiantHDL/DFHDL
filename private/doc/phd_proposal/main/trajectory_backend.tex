\chapter{Research Trajectory: Alternative Backends}
\label{chap:trajectory_backend}
Currently, DFiant designs are compiled solely into a single-clocked synchronous Xilinx FPGA. The limitation is not in the language, but in the preliminary backend implementation. In this trajectory, we implement different backend possibilities to truly prove the language's agnosticity, as well as exploring advantages of asynchronous logic design, multiple clock domains, and ASIC devices. Using the same design and comparing across such backend variance may prove to be an invaluable and unique tool trait.

\section{Multiple Clock Domains}
\label{sec:multiple_clock_domains}
Implementing multiple clock domains can save power, but for modern design flow it requires different synchronous RTL description (using another clock for some processes), explicit clock generators, different constraints, and possibly CDC snchronizers. The DFiant compiler backend can generate a muli-clocked RTL design without designer hassle. It is an open question how to properly select regions for optimal clock domains.

\section{Asynchronous Logic}
As we covered in the related work chapter, asynchronous logic design has been long possible, but infrequently applied. Since DFiant is essentially asynchronous at its core, it can also utilize the proper primitives to generate an asynchronous design, while comparing it against an equivalent synchronous design.

\section{ASIC Devices}
While FPGAs have specific structure, hard-IPs, and clock-trees, ASIC design is less constrained, but creates a different set of challenges we have yet faced.
