\section{Conclusion}
\label{sec:conclusion}
In this paper we proposed a novel DF-HDL abstraction layer that goes beyond RTL to free designs from being coupled to specific device or timing constraints. This abstraction layer tosses aside the register and wire RTL foundation blocks in favor of dataflow principles discussed in \sect{fig:motivation}. We also evolved the DFiant HDL and compiler to support dataflow state and many other useful constructs. DFiant provides a seamless concurrent programming approach, and yet it still facilitates a versatile compositional and hierarchical expressiveness. 

To evaluate the DFiant language and compiler, we reimplemented various RTL designs in DFiant and compared their performance, utilization, and LoCs. We demonstrated that most DFiant designs have equivalent performance and utilization to their RTL counterparts, yet manage to save between 50\% to 70\% LoCs. Evidently, the potential to increase designer productivity is significant, but notwithstanding, the greatest potential of DFiant is laid not in its concise syntax, but in its agnostic hardware description.  


Future work may explore dedicated simulation backend, an asynchronous logic backend, support for dynamic dataflow, or the missing abstractions like \emph{timers} (see \sect{fig:motivation}).
